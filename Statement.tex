\documentclass[dvips,12pt]{article}

\usepackage[pdftex]{graphicx}
\usepackage{url}


\setlength{\oddsidemargin}{0.25in}
\setlength{\textwidth}{6.5in}
\setlength{\topmargin}{0in}
\setlength{\textheight}{8.5in}


\begin{document}


\title{Sharing confidential information preserving the privacy}
\author{Luca Melis}
\date{\today}



\maketitle
\section{Motivations}
The word \emph{privacy} commonly refers to the ability of a person or a group to make information about themselves private. Governments, corporations or even private individuals may use wrongly or improperly sensitive information regarding a person. Indeed, many countries have laws and constitutions that in some way limit privacy.


In today's world, data is generally shared in a digital format and gathered by third parties.
Thus the leak of private information is becoming a critical concern involving forensic and financial  privacy consequences.
   
The need for privacy-preserving sharing of confidential information happens in many realistic situations.
A typical scenario may involve two entities: one that that is either motivated, or obliged, to share only the information desired from the other.
It is clear that in this case there is a strong contradiction between the need of information sharing and the need of privacy.

Let's consider the following examples:
\begin{itemize}
\item \textbf{Law authorities} Two law enforcement agencies having different lists of terrorist suspects are motivated to make a comparison between their respective databases. Nonetheless, They cannot unveil their entire databases without breaking national privacy laws. On the other hand, they can uncover only a list of potential terrorists of mutual interest.
\item \textbf{Genomics} Genome Sequencing is a process that determines the complete DNA sequence of an organism's genome at a single time. Biotechnology companies consider genetic test details as a valuable intellectual property and thus they want to unveil only the minimum amount of information required for the test.
\item \textbf{Face recognition} Automatic recognition of human faces techniques are employed in a wide range of application ranging from surveillance systems to social networks raising privacy concerns. In such a scenario, a system for privacy-preserving face recognition may involve two parties: a client searching for a human face in the database of a server .
\item \textbf{Internet security} Internet service providers (ISP) usually maintain a \emph{black-list} of potential attackers. Each ISP is motivated to know if others ISPs have some blacklist members in common without revealing the whole list.
\item \textbf{Multimedia File Similarity} Copyright concerns prevents two parties, willing to compute the similarity of their multimedia files, to share the actual content of these files. Privacy-preserving techniques may represent a valid solution to overcome this problem.
\end{itemize} 

The aforementioned examples motivate the need for techniques for sharing confidential information with privacy and present two interesting questions: 
\begin{enumerate}
\item how to permits privacy-preserving sharing of information allowing parties to learn only the strictly required information, 
\item how to do build efficient techniques for privacy-preserving sharing of information.
\end{enumerate} 


\section{Privacy-preserving cryptographic protocols}
Motivated by the privacy concerns introduced by modern technologies, recently a lot of research activities has been focused in the field of \emph{Privacy-Enhancing Technologies} (PETs).
Modern cryptography has contributed to PETs, producing several cryptographic protocols for privacy-preserving. 
A number of cryptographic protocols can be defined as the secure and privacy-preserving implementation of a public functionality $f$.

In the famous paradigm referred as \emph{Secure Multi-party Computation} (MPC) \cite{Yao}, two parties, willing to compute a function $f$ over their input, will only learn the output of $f$. This means that the two parties will not gain any further information about . 
The parties achieve this without the help of a trusted third party.
In the near future, MPC protocols may have a significant boost in their performances, thus allowing to use MPC in many realistic scenarios \cite{orlandi2011multiparty}.

My research objective will be focusing on the secure computation of specific functionalities, using specialized protocols rather than generic solutions such as those derived from MPC.
Two important motivations are:
\begin{enumerate}
\item it is sometimes difficult to use generic solutions for implementing all the functionality needed for sharing information,
\item ad-hoc protocols represent a more efficient way to do that.
\end{enumerate} 

In many real applications, parties request sensitive information from other entities, e.g., to retrieve messages, files or database records. 

The most famous mechanism to address this problem is Oblivious Transfer (OT)\cite{oblivious}.
OT permits a client to transfer one of potentially many messages to a server, remaining oblivious about the message transferred (if any).

Similar techniques are \emph{Private Information Retrieval} (PIR) \cite{pir} and \emph{Private Set Intersection} (PSI) \cite{psi}. 

In particular, PSI involves two parties, a server and a client, each with a private input set. PSI
lets parties run a cryptographic protocol that only disclose to the client the set intersection, and
nothing to the server (beyond client set size).
In prior works, some variants of PSI have been introduced:
\begin{itemize}
\item \emph{Private Set Intersection Cardinality} (PSI-CA) in which the client learns only the cardinality of the set intersection.
\item \emph{Authorized Private Set Intersection} (APSI) in which each item in client set must be authorized by some mutually trusted authority in order to ensure that client obtains duly authorized information.
\item \emph{Private Set Intersection with Data Transfer} (PSI-DT) in which the client also receives data records associated with each item in the intersection set.
\end{itemize}

In PSI, an interesting privacy property is that of hiding the size of the set held by one party from the other \cite{size-psi}.

In \cite{lin-psi,linear-psi}, the authors designed PSI protocols with linear computational complexities.

\subsection{Open Problems}
In recent years, there has been an increased interest in privacy-preserving protocols. 
Nonetheless there are some problems, both theoretical and practical, that are still open and that represents the main motivation of my PhD work.
\subsubsection{Efficient Group Private Set Intersection}
The traditional PSI formulation only includes two participants, server and client. However, it is not clear how to efficiently extend such techniques to scenarios where a group of n participants (with $n > 2$) wish to privately compute the intersection of their respective sets (without using a trusted or semi-trusted party). Multi-party PSI protocols with linear complexities still remains a challenging topic for further research.
\subsubsection{Size-Hiding Private Set Intersection Secure in Malicious Model}
One important factor on the security of cryptographic is the adversarial model which is either semi-honest or malicious. Protocols secure against \emph{semi-honest adversaries} assume that participants strictly follow the steps of the protocols but they might attempt to infer additional information about other party's input. On the other hand, security against \emph{malicious adversaries} permits arbitrary deviations from the protocols.

In \cite{size-psi}, the PSI with the \emph{size-hiding} property, is provably secure under standard assumptions only against semi-honest adversaries. Future research is needed in order to design a size-hiding PSI secure against malicious adversaries.  
\subsubsection{Privacy in testing genomic information}
In \cite{genome}, the authors show how to enable privacy in genomic applications, e.g., paternity tests, genetic and personalized medicine testing. A further step in that direction is to implement specific cryptographic tools for genomic privacy protection.

\section{Personal background}
My background is mainly that of a Computer Engineer.
I'm able to write programs using the most important programming languages such as $C / C++$, Python, Java and Matlab/Octave.

During my studies, I developed analytical skills and learnt how to deal with complex problems in a more systematic way. Some of the courses I followed turned out to be very relevant for the field of cryptography and information technology security. 
In particular, I took the courses of cryptography and number theory at the faculty of mathematics and I became familiar with the fundamentals of Cryptography and abstract mathematics.
At the faculty of computer engineering I took some courses of applied cryptography such as Network Security and Security of Multimedia Contents.

In the practical realisation of my bachelor's degree thesis, I implemented a prominent cryptographic technique in the field of Multimedia contents security called \emph{Buyer-Seller protocol} \cite{tesi}. 
The results of my implementation has been mentioned in \cite{tesi2}.
I believe that my thesis has contributed to the improvement of my skills in the practical realization of  cryptographic protocols. These techniques also represent the main building blocks of the privacy-preserving protocols that constitute the subject of my PhD work.

During my stay at the University of Aarhus, I worked on my master's degree thesis about the cryptographic problem of \emph{Learning Parity with Noise} (LPN) \cite{lpn} under the supervision of Ivan Damg{\aa}rd and Claudio Orlandi. 
This thesis focused on theoretical aspects of cryptography. In this work I propose new cryptographic protocols derived from LPN addressing their correctness and security issues.

Finally, I would like to ask funding through departmental grants for giving me the opportunity to work closely and collaboratively with mentors and other graduate students on the long-term projects mentioned in this research proposal.
\begin{thebibliography}{99}

\bibitem{Yao}
	A. Yao,
	Protocols for secure computations,	
	In FOCS, pages 160–164, 1982.

\bibitem{orlandi2011multiparty}
	Orlandi, Claudio,  	
  	Is multiparty computation any good in practice?,
  	{Acoustics, Speech and Signal Processing (ICASSP), 2011 IEEE International Conference on},
  	2011,
  	IEEE.

\bibitem{pir}
	B. Chor, E. Kushilevitz, O. Goldreich, and M. Sudan,	
	Private information retrieval,
	Journal of ACM, 45(6):965–981, 
	1998.

\bibitem{oblivious}
	M. Rabin,
	How to exchange secrets by oblivious transfer,
	TR-81, Harvard Aiken Computation Lab, 
	1981.

\bibitem{psi}
	Emiliano De Cristofaro and Paolo Gasti and Gene Tsudik,	
	Fast and Private Computation of Cardinality of Set Intersection and Union, 
    Cryptology ePrint Archive, Report 2011/141,
    2011.
    
\bibitem{size-psi}
	Giuseppe Ateniese and Emiliano De Cristofaro and Gene Tsudik,
    (If) Size Matters: Size-Hiding Private Set Intersection,
    Cryptology ePrint Archive, Report 2010/220,
	2010.

\bibitem{lin-psi}
	Emiliano De Cristofaro and Jihye Kim and Gene Tsudik,	
	Linear-Complexity Private Set Intersection Protocols Secure in Malicious Model,
	Cryptology ePrint Archive, Report 2010/469,
	2010.

\bibitem{linear-psi}
	Changyu Dong and Liqun Chen and Zikai Wen,
	When Private Set Intersection Meets Big Data: An Efficient and Scalable Protocol,
	Cryptology ePrint Archive, Report 2013/515,
	2013.
	
\bibitem{genome}
	P. Baldi, R. Baronio, E. De Cristofaro, P. Gasti, and G. Tsudik. 
	Countering GATTACA: Efficient and Secure Testing of Fully-Sequenced Human Genomes.
	In CCS, 
	2011.
	
\bibitem{tesi}
	Deng, Mina and Bianchi, Tiziano and Piva, Alessandro and Preneel, Bart,
	An Efficient Buyer-seller Watermarking Protocol Based on Composite Signal Representation,
	Proceedings of the 11th ACM Workshop on Multimedia and Security,
	2009.
	
\bibitem{tesi2}
	Rial, A. and Mina Deng and Bianchi, T. and Piva, A. and Preneel, B.
	A Provably Secure Anonymous Buyer; Seller Watermarking Protocol,
	Information Forensics and Security, IEEE Transactions on,
	2010	.	

\bibitem{lpn}	
	Krzysztof Pietrzak. 
	Cryptography from learning parity with noise.
	In SOFSEM 2012: Theory and Practice of Computer Science,
	2012.
	

\end{thebibliography}


\end{document}


